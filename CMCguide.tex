\documentclass[11pt,twoside,space]{article}
\usepackage{CMC}
\title{第九届中国大学生数学竞赛预赛试卷}
\author{14 金融工程--零蛋大}
\date{2017年10月28日}
\type{数学类}
\examtime{150}
\watermark{48}{11}{机密}
%\usepackage{blindtext}
\begin{document}
\maketitle
\begin{center}
\zihao{-4}
\begin{tabular}{|m{3em}<{\centering}|*{7}{m{3.5em}<{\centering}|}}
\hline
题~号 & 一 & 二 & 三  & 四 & 五 & 六  &总~~分 \\
\hline
满~分 & 15 & 15 & 15  & 20 & 15 & 20  &\raisebox{0.38em}{100}\rule{0pt}{8mm}\\
\hline
得~分 &    &    &     &    &    &     &\rule{0pt}{8mm}\\
\hline	
\end{tabular}\vspace*{0.6em}		
$\begin{aligned}
\mbox{注意:}
&1.\,\mbox{所有答题都须写在试卷密封线右边,写在其他纸上一律无效}.\hspace{12.0cm}\\
&2.\,\mbox{密封线左边请勿答题,密封线外不得有姓名及相关标记}.\\
&3.\,\mbox{如答题空白不够,可写在当页背面,并标明题号}.\\[-2mm]
\end{aligned}$	
\end{center}
\setlength{\marginparsep}{-0.8cm}
%%==================================================================
%%—————————————————————————————正文开始———————————————————————————————
%%==================================================================


\section{表格}
官方的题数无规律,只能单独设置

\section{中文字体说明}
adobe 宋体(默认),adobe 黑体\verb|\|heiti,adobe 仿宋\verb|\|fangsong,adobe 楷体\verb|\|kaishu,华文行楷\verb|\|xingkai,华文中宋\verb|\|zhongsong


弃方正字体

\section{英文字体说明}
英文字体粗斜:推荐使用 \verb|$\bm{abc}$|$\bm{abc}$,不推荐\verb|\|bfitTimes New Roman Bold Italic


加入数学字体mtpro2宏包,mtpro2宏包的安装参考 LaTeX技巧693:安装 MathTime Professional 2 数学字体\url{http://www.latexstudio.net/archives/241.html}


弃Times New Roman 风格

\section{解答环境}
Proof,Solution

\verb|\dfsxian{得分}|,\verb|\cfsxian{得分}|,\verb|\fsxian{长度(cm)}{得分}|

\dfsxian{9}
\cfsxian{10}
\fsxian{1}{10}

\section{大题}
\verb|\section{大题}|, \verb|\dati{}{}| 二选一\par
\dati{}{(本题15分)\;\;%\\[2mm]
	在空间直角坐标系中,设单叶双曲面 $\Gamma$ 的方程为 $x^2+y^2-z^2=1$,设 $P$ 为空间的平面, 它交 $\Gamma$ 于一抛物线 $C$. 求该平面的法线与 $z$- 轴的夹角.\\} 

\section{选择题}
设可导函数 $f(x)$ 的原函数是 $F(x)$, 可导函数 $g(x)$ 的原函数是 $G(x)$, $g(x)$ 是 $f(x)$ 在区间 $I$ 上的反函数,  则\hfill(\qquad)\\[-1.2em]
\begin{tasks}(2) % 1,2,4
\task $F'(x)G'(x)=1$
\task $f'(x)g'\big(f(x)\big)=1$
\task $\frac{\dif G\big(f(x)\big)}{\dif x}=-1$
\task $\frac{\dif F\big(g(x)\big)}{\dif x}=1$	
\end{tasks}

\section{一面双页}
先保存再编译,1-6 是从第一页到第六页
\begin{verbatim*}
\documentclass{article}
\usepackage{pdfpages}
\usepackage[paperwidth=40cm,paperheight=27.5cm]{geometry}
\begin{document}
\includepdf[pages=1-6,nup=2x1,scale=1,
offset=3mm 0mm,column,delta=-10 -0mm]{17mathSJ.pdf}	
\end{document}
\end{verbatim*}
见末尾。



%%%%%%%%%%%%%%%%%%%%%%%%%%%%%%%%%%%%%%%%%%%%%%%%%%%%%%%%%%%%%%%%%%%%%
%------------------------------------结束--------------------------------------
%%%%%%%%%%%%%%%%%%%%%%%%%%%%%%%%%%%%%%%%%%%%%%%%%%%%%%%%%%%%%%%%%%%%%
\clearpage
\end{document}