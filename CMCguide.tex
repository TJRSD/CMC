\documentclass[11pt,twoside]{article} 
\usepackage[mtpro2]{CMC}
\usepackage{bm}
\title{第十一届中国大学生数学竞赛预赛参考答案}
\author{14 金融工程--零蛋大}
\date{2019年10月27日}
\type{非数学类}
\examtime{150}
%\watermark{48}{11}{机密}
%\renewenvironment{solution}{\setbox0\vbox\bgroup}{\egroup\unskip}
\usepackage{listings}
% settings for listings.sty
% https://www.latexstudio.net/archives/8420.html
%\usepackage[usenames, dvipsnames]{xcolor}
\definecolor{ForestGreen}{rgb}{0.0,0.34,0.0}
\definecolor{frenchplum}{RGB}{129,20,83}
\renewcommand{\lstlistingname}{代码清单}
\lstdefinestyle{lfonts}{
  basicstyle   = \footnotesize\ttfamily,
  stringstyle  = \color{purple},
  keywordstyle = \color{blue!60!black}\bfseries,
  commentstyle = \color{olive}\normalfont,
}
\lstdefinestyle{lnumbers}{
  numbers     = left,
  numberstyle = \tiny,
  numbersep   = 1em,
  firstnumber = 1,
  stepnumber  = 1,
}
\lstdefinestyle{llayout}{
  breaklines       = true,
  tabsize          = 2,
  columns          = flexible,
}
\lstdefinestyle{lgeometry}{
  xleftmargin      = 20pt,
  xrightmargin     = 0pt,
  frame            = tb,
  framesep         = \fboxsep,
  framexleftmargin = 20pt,
}
\lstdefinestyle{lgeneral}{
  style = lfonts,
  style = lnumbers,
  style = llayout,
  style = lgeometry,
}
\def\beginlstdelim#1#2#3{%
  \def\endlstdelim{#2\egroup}%
  \ttfamily#1\bgroup\color{#3}\aftergroup\endlstdelim}
\lstdefinestyle{ldelims}{
  moredelim = **[is][\beginlstdelim{\$}{\$}{orange}]{\$}{\$},
  moredelim = **[is][\beginlstdelim{\{}{\}}{ForestGreen}]{\{}{\}},
  moredelim = **[is][\beginlstdelim{[}{]}{cyan}]{[}{]},
}
% LaTeX lst style
\lstdefinestyle{lltx}{
  language = {[LaTeX]TeX},
  style = lgeneral,
  style = ldelims,
  morekeywords = {% LaTeX original commands
    maketitle,
    xingkai,kaishu,songti,heiti,fangsong,zhongsong,
    rmfamily, sffamily, ttfamily,
    itshape, slshape, scshape,
    mdseries, bfseries, emph,
    textrm, textsf, texttt,
    textit, textsl, textsc,
    textmd, textbf,bfit,
    newcommand, renewcommand, providecommand,
    cs, meta, marg, oarg, parg,
    defen,score
  }
}
\lstdefinestyle{tsdtex}{
	language = {[LaTeX]TeX},
	style = lfonts,
	style = llayout,
	%style = lgeometry,
	style = ldelims,
	breaklines=true,
	numbers=none,
	morekeywords = {% LaTeX original commands
		maketitle,
		xingkai,kaishu,songti,heiti,fangsong,zhongsong,
		rmfamily, sffamily, ttfamily,
		itshape, slshape, scshape,
		mdseries, bfseries, emph,
		textrm, textsf, texttt,
		textit, textsl, textsc,
		textmd, textbf,bfit,
		newcommand, renewcommand, providecommand,
		cs, meta, marg, oarg, parg,
		defen,score
	}
}
\lstdefinestyle{iltx}{
  style      = lltx,
  basicstyle = \ttfamily
}
\lstdefinestyle{lbash}{
  language   = {bash},
  style      = lgeneral,
}
\lstdefinestyle{ibash}{
  style      = lbash,
  basicstyle = \ttfamily
}
\definecolor{winered}{rgb}{0.5,0,0}
\definecolor{structurecolor}{RGB}{60,113,183}
\lstset{language=[LaTeX]TeX,
	texcsstyle=*\color{winered},
	numbers=none,
	breaklines=true,
	keywordstyle=\color{winered},
	commentstyle=\color{gray},
	emph={elegantpaper,fontenc,fontspec,xeCJK,FiraMono,xunicode,newtxmath,figure,fig,image,img,table,itemize,enumerate,newtxtext,newtxtt,ctex,microtype,description,times,newtx,booktabs,tabular,PDFLaTeX,XeLaTeX,type1cm,BibTeX,device,color,mode,lang,amsthm,tcolorbox,titlestyle,cite,marginnote,ctex,listings},
	emphstyle={\color{frenchplum}},
	morekeywords={DeclareSymbolFont,SetSymbolFont,toprule,midrule,bottomrule,institute,version,includegraphics,setmainfont,setsansfont,setmonofont ,setCJKmainfont,setCJKsansfont,setCJKmonofont,RequirePackage,figref,tabref,email,maketitle,keywords,definecolor,extrainfo,logo,cover,subtitle,appendix,chapter,hypersetup,mainmatter,tableofcontents,elegantpar,numbers,authoryear,heiti,kaishu,lstset,pagecolor,zhnumber,marginpar,part,equote},
	frame=single,
	tabsize=2,
	rulecolor=\color{structurecolor},
	framerule=0.2pt,
	columns=flexible,
	% backgroundcolor=\color{lightgrey}
}
\endinput

%\lstinline[style=iltx]|命令|
\begin{document}
\maketitle
\begin{flushleft}
\zihao{-4}
\begin{tabular}{|m{2.8em}<{\centering}|*{7}{m{(0.78\textwidth-2.8em)/7}<{\centering}|}}
\hline
题号 & 一 & 二 & 三   & 四 & 五 & 六  & 总~~分 \\\hline
满分 & 30 & 14 & 14  & 14 & 14  & 14 & \raisebox{0.4em}{100}\rule{0pt}{8mm}\\
\hline
得分 &    &    &     &    &    &    &\rule{0pt}{8mm} \\
\hline		
\end{tabular}\vspace*{0.6em}		
$\begin{aligned}
\mbox{注意:}
&\text{本试卷共六大题, 满分100分, 考试时间为150分钟.}\\
&1.\,\mbox{所有答题都须写在试卷密封线右边,写在其他纸上一律无效}.\\
&2.\,\mbox{密封线左边请勿答题,密封线外不得有姓名及相关标记}.\\
&3.\,\mbox{如答题空白不够,可写在当页背面,并标明题号}.
\end{aligned}$		
\end{flushleft}
\setlength{\marginparsep}{-0.8cm}
%%==================================================================
%%—————————————————————————————正文开始———————————————————————————————
%%==================================================================		

模板基于TeX Live 2020

\[\int_0^x f(x) \dif x\]

\section{表格}
官方的题数无规律,只能单独设置
\begin{lstlisting}[style=tsdtex]
\begin{tabular}{|m{2.8em}<{\centering}|*{7}{m{(0.78\textwidth-2.8em)/7}<{\centering}|}}
\hline
题号 & 一 & 二 & 三   & 四 & 五 & 六  & 总~~分 \\\hline
满分 & 30 & 14 & 14  & 14 & 14  & 14 & \raisebox{0.4em}{100}\rule{0pt}{8mm}\\
\hline
得分 &    &    &     &    &    &    &\rule{0pt}{8mm} \\
\hline		
\end{tabular}
\end{lstlisting}

\begin{lstlisting}[style=tsdtex]
\begin{tabular}{|m{2.8em}<{\centering}|*{8}{m{(0.78\textwidth-2.8em)/8}<{\centering}|}}
\hline
题号 & 一 & 二 & 三  & 四 & 五 & 六 & 七 & 总~分 \\\hline
满分 & 30 & 10 & 10  & 12 & 12  & 12 & 14 & \raisebox{0.4em}{100}\rule{0pt}{8mm}\\
\hline
得分 &    &    &     &    &    &    &    &\rule{0pt}{8mm} \\
\hline		
\end{tabular}
\end{lstlisting}

\section{中文字体说明}

默认中文字体为思源 Noto 系列
\begin{itemize}
\item 安装字体的正确姿势:\textbf{鼠标右键 $\rightarrow$ 为所有用户安装(A)}
\item 字体下载:\url{https://share.weiyun.com/9SuIV2YB}
\item 自定义字体命令:{\cusong 思源宋体Bold}: \lstinline[style=iltx]|{\cusong 思源宋体Bold}|
\end{itemize}

%\begin{enumerate}
%\item abobe字体
%\item 自定义:\lstinline[style=iltx]|{\zhongsong 华文中宋}| {\zhongsong 华文中宋}、\lstinline[style=iltx]|{\xingkai 华文行楷}| {\xingkai 华文行楷}
%\end{enumerate}

%\begin{table}[htbp]
%\caption{中文字体说明}
%\centering
%\begin{tabular}{|cll|}
%\hline 
%\songti adobe宋体    & Adobe Song Std L     & \lstinline[style=iltx]|\songti adobe宋体|    \\
%\kaishu adobe楷体    & Adobe Kaiti Std R    & \lstinline[style=iltx]|\kaishu adobe楷体|    \\
%\heiti adobe黑体     & Adobe Heiti Std R    & \lstinline[style=iltx]|\heiti adobe黑体|     \\
%\fangsong adobe仿宋  & Adobe Fangsong Std R & \lstinline[style=iltx]|\fangsong adobe仿宋|  \\
%\zhongsong 华文中宋  & STZhongsong          & \lstinline[style=iltx]|\zhongsong 华文中宋|   \\
%\xingkai 华文行楷    & STXingkai            &  \lstinline[style=iltx]|\xingkai 华文行楷|    \\
%\hline	
%\end{tabular} 
%\end{table}

\section{英文字体说明}

\begin{enumerate}
\item 英文字体粗斜:
\begin{enumerate}
	\item 推荐bm宏包(默认无) \lstinline[style=iltx]|$\bm{abcABC}$|$\bm{abcABC}$,\lstinline[style=iltx]|$\mbf{abcABC}$|$\mbf{abcABC}$
	\item 不推荐 \lstinline[style=iltx]|\bfit{Times New Roman Bold Italic}| \bfit{Times New Roman Bold Italic}
\end{enumerate}
\item 是否使用 mtpro2字体宏包?(非默认数学字体)
\begin{enumerate}
\item 使用mtpro2字体
\begin{itemize}
\item 安装:\href{http://www.latexstudio.net/archives/241.html}{LaTeX技巧693:安装 MathTime Professional 2 数学字体}
\item 设置:math=mtpro2, 需安装 mtpro2 字体
\end{itemize} 
\item 不使用mtpro2字体
\begin{itemize}
\item 默认字体(无需安装), 也可选参数math=cm(默认无需添加)
\end{itemize}
\end{enumerate}
\item 已弃 Times New Roman 风格
\end{enumerate}

\section{填空题}\label{sec:tiankongti}
\lstinline[style=iltx]|\blank{填空题答案}| \blank{填空题答案} 
\begin{lstlisting}
%\renewenvironment{solution}{\setbox0\vbox\bgroup}{\egroup\unskip} %注释则不显示solution环境的内容,即不显示填空题答案
\begin{solution}
填空题的解答环境
\end{solution}
\end{lstlisting}

\section{大题}
\lstinline[style = iltx]|\section{大题}|和
\lstinline[style=iltx]|\dati[默认缩进2em]{大题}|
\begin{lstlisting}[style=tsdtex]
\dati{(本题15分) 在空间直角坐标系中,设单叶双曲面 $\Gamma$ 的方程为 $x^2+y^2-z^2=1$,设 $P$ 为空间的平面, 它交 $\Gamma$ 于一抛物线 $C$. 求该平面的法线与 $z$- 轴的夹角}
\end{lstlisting}
\dati{(本题15分) 在空间直角坐标系中,设单叶双曲面 $\Gamma$ 的方程为 $x^2+y^2-z^2=1$,设 $P$ 为空间的平面, 它交 $\Gamma$ 于一抛物线 $C$. 求该平面的法线与 $z$- 轴的夹角}\vspace{1ex}

\begin{lstlisting}[style=tsdtex]
\dati[0em]{(本题15分) 在空间直角坐标系中,设单叶双曲面 $\Gamma$ 的方程为 $x^2+y^2-z^2=1$,设 $P$ 为空间的平面, 它交 $\Gamma$ 于一抛物线 $C$. 求该平面的法线与 $z$- 轴的夹角}
\end{lstlisting}
\dati[0em]{(本题15分) 在空间直角坐标系中,设单叶双曲面 $\Gamma$ 的方程为 $x^2+y^2-z^2=1$,设 $P$ 为空间的平面, 它交 $\Gamma$ 于一抛物线 $C$. 求该平面的法线与 $z$- 轴的夹角}\vspace{1ex}

\vspace{2.5ex}
\noindent 解答环境: Solution, Proof, proof
%\begin{lstlisting}[style=tsdtex]
%\begin{Solution}
%Solution 环境
%\end{Solution}
%\end{lstlisting}
\begin{Solution}
Solution 环境
\end{Solution}

%\begin{lstlisting}[style=tsdtex]
%\begin{proof}
%proof 环境
%\end{proof}
%\end{lstlisting}
\begin{proof}
proof 环境
\end{proof}

%\begin{lstlisting}[style=tsdtex]
%\begin{Proof}
%Proof 环境
%\end{Proof}
%\end{lstlisting}
\begin{Proof}
Proof 环境
\end{Proof}

\vspace{1.5em}
\noindent 分数线
\begin{itemize}
\item 推荐 \lstinline[style=iltx]|\defen[默认0cm]{得分}|、\lstinline[style=iltx]|\score[默认0cm]{得分}|
\item 不推荐 \lstinline[style=iltx]|\dfsxian{得分}|,\lstinline[style=iltx]|\cfsxian{得分}|,\lstinline[style=iltx]|\fsxian{长度(cm)}{得分}|
\end{itemize}



\section{选择题}
\begin{lstlisting}[style=tsdtex]
\begin{tasks}(2) % 1,2,4
\task $F'(x)G'(x)=1$
\task $f'(x)g'\big(f(x)\big)=1$
\task $\frac{\dif G\big(f(x)\big)}{\dif x}=-1$
\task $\frac{\dif F\big(g(x)\big)}{\dif x}=1$	
\end{tasks}
\end{lstlisting}
\begin{tasks}(2) % 1,2,4
\task $F'(x)G'(x)=1$
\task $f'(x)g'\big(f(x)\big)=1$
\task $\frac{\dif G\big(f(x)\big)}{\dif x}=-1$
\task $\frac{\dif F\big(g(x)\big)}{\dif x}=1$	
\end{tasks}


\section{答案与试卷分离}

默认显示全部答案, 模板提供了可选参数 \lstinline[style=iltx]|result=answer|以及\lstinline[style=iltx]|result=noanswer|
\begin{enumerate}
\item \lstinline[style=iltx]|result=answer| 显示试卷以及全部答案(包括选择题解析)
\item \lstinline[style=iltx]|result=noanswer| 只显示试卷
\item 填空题解析是否隐藏见: \ref{sec:tiankongti}
\end{enumerate}  

\section{自定义命令}
\begin{enumerate}
\item 微分算子d: \lstinline[style=iltx]|\dif|
\item 矩阵的秩rank: \lstinline[style=iltx]|\rank|
\end{enumerate}

%%%%%%%%%%%%%%%%%%%%%%%%%%%%%%%%%%%%%%%%%%%%%%%%%%%%%%%%%%%%%%%%%%%%%
%-------------------------------结束---------------------------------
%%%%%%%%%%%%%%%%%%%%%%%%%%%%%%%%%%%%%%%%%%%%%%%%%%%%%%%%%%%%%%%%%%%%%
\clearpage
\end{document}
\section{一面双页}

先保存再编译,1-6 是从第一页到第六页
\begin{verbatim}
\documentclass{article}
\usepackage{pdfpages}
\usepackage[paperwidth=40cm,paperheight=27.5cm]{geometry}
\begin{document}
\includepdf[pages=1-6,nup=2x1,scale=1,
offset=3mm 0mm,column,delta=-10 -0mm]{17mathSJ.pdf}	
\end{document}
\end{verbatim}
